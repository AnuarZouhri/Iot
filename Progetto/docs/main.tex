\documentclass[12pt,a4paper]{article}

% --- Pacchetti utili ---
\usepackage[utf8]{inputenc}
\usepackage[T1]{fontenc}
\usepackage[italian]{babel}
\usepackage{amsmath, amssymb}
\usepackage{graphicx}
\usepackage{hyperref}
\usepackage{listings}
\usepackage{xcolor}
\usepackage{caption}
\usepackage{subcaption}
\usepackage{fancyhdr}
\usepackage{geometry}
\geometry{margin=2.5cm}

% --- Impostazioni per il codice sorgente ---
\definecolor{codebg}{rgb}{0.95,0.95,0.95}
\lstset{
	backgroundcolor=\color{codebg},
	basicstyle=\ttfamily\small,
	keywordstyle=\color{blue}\bfseries,
	commentstyle=\color{gray}\itshape,
	stringstyle=\color{red},
	numbers=left,
	numberstyle=\tiny,
	numbersep=5pt,
	frame=single,
	breaklines=true,
	captionpos=b
}

% --- Header & Footer ---
\pagestyle{fancy}
\fancyhf{}
\fancyhead[L]{Documentazione Progetto IoT}
\fancyhead[R]{\leftmark}
\fancyfoot[C]{\thepage}

% --- Inizio documento ---
\begin{document}
	
	\begin{titlepage}
		\centering
		{\scshape\LARGE Università degli Studi di Salerno \par}
		\vspace{1cm}
		{\scshape\Large Corso di Laurea in Ingegneria Informatica\par}
		\vspace{2cm}
		{\huge\bfseries Documentazione Progetto IoT\par}
		\vspace{1.5cm}
		{\Large Annamaria Scermino, Anuar Zourhi, Gerardo Selce \par}
		\vspace{1.5cm}
		{\huge\bfseries TheBox\par}
		\vfill
		{\large \today\par}
	\end{titlepage}
	
	\tableofcontents
	\newpage
	
	\section{Introduzione}
	Il progetto sviluppato ha come obiettivo la realizzazione di uno smart caveaux. Il caveaux è dotato di un meccanismo di apertura che può essere controllato sia tramite un tastierino fisico posto all’esterno, sia da remoto attraverso una piattaforma di controllo. Il sistema integra inoltre sensori per la rilevazione di temperatura e umidità, con la capacità di attivare allarmi in caso di valori fuori soglia o tentativi di intrusione. Gli allarmi possono essere gestiti localmente o a distanza. Un semaforo posto accanto alla porta mostra lo stato del caveaux.
	
	\section{Architettura del Sistema}
	I componenti hardware utilizzati sono i seguenti:
	\begin{itemize}
		\item Un servomotore;
		\item Un tastierino numerico;
		\item Uno schermo OLED;
		\item Due breadboard;
		\item Due push-button;
		\item Un semaforo;
		\item Un sensore ad ultrasuoni;
		\item Un sensore di umidità e temperatura;
		\item Un buzzer passivo;
		\item Una ESP32
		\item Un relay 
	\end{itemize}
	Dal punto di vista software, il sistema è stato sviluppato interamente in micropython. Per il controllo remoto e il monitoraggio in tempo reale è stata realizzata una dashboard interattiva con Node-RED, che consente di visualizzare lo stato del sistema, ricevere notifiche e inviare comandi. Inoltre alcune funzionalità, come l'apertura del caveaux, sono accessibili anche da un'app mobile. I protocolli utilizzati sono: MQTT ed I2C.
	
	\section{Descrizione del Codice}
	Spiegazione dettagliata del codice, divisa per moduli o file.
	
	\subsection{File: \texttt{main.ino}}
	\begin{lstlisting}[language=C++, caption={File principale per ESP32}]
		#include <WiFi.h>
		#include <PubSubClient.h>
		
		// Connessione WiFi
		const char* ssid = "NomeRete";
		const char* password = "password";
		
		// Setup
		void setup() {
			Serial.begin(115200);
			WiFi.begin(ssid, password);
			while (WiFi.status() != WL_CONNECTED) {
				delay(500);
				Serial.print(".");
			}
			Serial.println("Connesso!");
		}
	\end{lstlisting}
	
	
	
\end{document}
