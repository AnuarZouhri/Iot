\documentclass[12pt,a4paper]{article}

% --- Pacchetti utili ---
\usepackage[utf8]{inputenc}
\usepackage[T1]{fontenc}
\usepackage[italian]{babel}
\usepackage{amsmath, amssymb}
\usepackage{graphicx}
\usepackage{hyperref}
\usepackage{listings}
\usepackage{xcolor}
\usepackage{caption}
\usepackage{subcaption}
\usepackage{fancyhdr}
\usepackage{geometry}
\geometry{margin=2.5cm}

% --- Impostazioni per il codice sorgente ---
\definecolor{codebg}{rgb}{0.95,0.95,0.95}
\lstset{
	backgroundcolor=\color{codebg},
	basicstyle=\ttfamily\small,
	keywordstyle=\color{blue}\bfseries,
	commentstyle=\color{gray}\itshape,
	stringstyle=\color{red},
	numbers=left,
	numberstyle=\tiny,
	numbersep=5pt,
	frame=single,
	breaklines=true,
	captionpos=b
}

% --- Header & Footer ---
\pagestyle{fancy}
\fancyhf{}
\fancyhead[L]{Documentazione Progetto IoT}
\fancyhead[R]{\leftmark}
\fancyfoot[C]{\thepage}

% --- Inizio documento ---
\begin{document}
	
	\begin{titlepage}
		\centering
		{\scshape\LARGE Università degli Studi di Salerno \par}
		\vspace{1cm}
		{\scshape\Large Corso di Laurea in Ingegneria Informatica\par}
		\vspace{2cm}
		{\huge\bfseries Documentazione Progetto IoT\par}
		\vspace{1.5cm}
		{\Large Annamaria Scermino, Anuar Zourhi, Gerardo Selce \par}
		\vspace{1.5cm}
		{\huge\bfseries TheBox\par}
		\vfill
		{\large \today\par}
	\end{titlepage}
	
	\tableofcontents
	\newpage
	
	\section{Introduzione}
	Descrizione generale del progetto: obiettivi, contesto, tecnologie usate (es. ESP32, MQTT, Node-RED, ecc.).\\ \\
	
	E' stato progettato uno smart caveaux
	
	\section{Architettura del Sistema}
	Descrizione dei componenti hardware e software. Possibile diagramma a blocchi o schema circuitale:
	
	
	\section{Descrizione del Codice}
	Spiegazione dettagliata del codice, divisa per moduli o file.
	
	\subsection{File: \texttt{main.ino}}
	\begin{lstlisting}[language=C++, caption={File principale per ESP32}]
		#include <WiFi.h>
		#include <PubSubClient.h>
		
		// Connessione WiFi
		const char* ssid = "NomeRete";
		const char* password = "password";
		
		// Setup
		void setup() {
			Serial.begin(115200);
			WiFi.begin(ssid, password);
			while (WiFi.status() != WL_CONNECTED) {
				delay(500);
				Serial.print(".");
			}
			Serial.println("Connesso!");
		}
	\end{lstlisting}
	
	\subsection{Funzionamento MQTT}
	Spiegazione su come vengono pubblicati/sottoscritti i messaggi.
	
	\section{Interfaccia Utente}
	Breve descrizione di eventuale dashboard (es. in Node-RED, Blynk, o web app).
	
	\section{Risultati e Test}
	Dati rilevati, log di funzionamento, grafici, considerazioni.
	
	\section{Conclusioni}
	Riflessioni finali, limiti, possibili miglioramenti.
	
	\section*{Allegati}
	\begin{itemize}
		\item Codice completo su GitHub: \url{https://github.com/tuo_repo}
		\item Diagrammi circuitali
		\item Link a demo o video
	\end{itemize}
	
\end{document}
